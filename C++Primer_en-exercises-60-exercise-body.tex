% ------------------------------------------------------------------------
% file `C++Primer_en-exercises-60-exercise-body.tex'
%
%     exercise of type `exercises' with id `60'
%
% generated by the `exercises' environment of the
%   `xsim' package v0.8a (2017/05/19)
% from source `C++Primer_en' on 2017/12/05 on line 9684
% ------------------------------------------------------------------------
\begin{question}
Write a program using a series of \verb|if| statements to count the
number of vowels in text read from \verb|cin|.
\end{question}

\begin{question}
There is one problem with our vowel-counting program as
we’ve implemented it: It doesn’t count capital letters as vowels. Write a
program that counts both lower- and uppercase letters as the appropriate
vowel—that is, your program should count both \verb|'a'| and \verb|'A'| as part of
\verb|aCnt|, and so forth.
\end{question}

\begin{question}
Modify our vowel-counting program so that it also counts the
number of blank spaces, tabs, and newlines read.
\end{question}

\begin{question}
Modify our vowel-counting program so that it counts the
number of occurrences of the following two-character sequences: \verb|ff|, \verb|fl|,
and \verb|fi|.
\end{question}

\begin{question}
Each of the programs in the highlighted text on page \pageref{lst:code for exercises 5-13}
contains a common programming error. Identify and correct each error.
\begin{enumerate}[label=(\alph*)]
^^I\item
\begin{lstlisting}[label={lst:code for exercises 5-13}]
unsigned aCnt = 0, eCnt = 0, iouCnt = 0;
char ch = next_text();
   switch (ch) {
      case 'a': aCnt++;
      case 'e': eCnt++;
      default: iouCnt++;
    }
\end{lstlisting}

^^I\item
\begin{lstlisting}
unsigned index = some_value();
switch (index) {
   case 1:
      int ix = get_value();
      ivec[ ix ] = index;
      break;
   default:
      ix = ivec.size()-1;
      ivec[ ix ] = index;
}
\end{lstlisting}

^^I\item
\begin{lstlisting}
unsigned evenCnt = 0, oddCnt = 0;
int digit = get_num() % 10;
switch (digit) {
   case 1, 3, 5, 7, 9:
      oddcnt++;
      break;
   case 2, 4, 6, 8, 10:
      evencnt++;
      break;
}
\end{lstlisting}

^^I\item
\begin{lstlisting}
unsigned ival=512, jval=1024, kval=4096;
unsigned bufsize;
unsigned swt = get_bufCnt();
switch(swt) {
   case ival:
      bufsize = ival * sizeof(int);
      break;
   case jval:
      bufsize = jval * sizeof(int);
      break;
   case kval:
      bufsize = kval * sizeof(int);
      break;
}
\end{lstlisting}

\end{enumerate}
\end{question}
