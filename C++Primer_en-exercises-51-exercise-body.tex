% ------------------------------------------------------------------------
% file `C++Primer_en-exercises-51-exercise-body.tex'
%
%     exercise of type `exercises' with id `51'
%
% generated by the `exercises' environment of the
%   `xsim' package v0.8a (2017/05/19)
% from source `C++Primer_en' on 2017/12/05 on line 8076
% ------------------------------------------------------------------------
\begin{question}
Write a program to use a conditional operator to find the
elements in a \verb|vector<int>| that have odd value and double the value of
each such element.
\end{question}

\begin{question}
Extend the program that assigned high pass, pass, and fail
grades to also assign low pass for grades between 60 and 75 inclusive. Write
two versions: One version that uses only conditional operators; the other
should use one or more \verb|if| statements. Which version do you think is easier
to understand and why?
\end{question}

\begin{question}
The following expression fails to compile due to operator
precedence. Using Table \ref{tab:operator precedence}  (p. \pageref{tab:operator precedence} ), explain why it fails. How would you
fix it?
\begin{lstlisting}
string s = "word";
string pl = s + s[s.size() - 1] == 's' ? "" : "s" ;
\end{lstlisting}
\end{question}

\begin{question}
Our program that distinguished between high pass, pass,
and fail depended on the fact that the conditional operator is right
associative. Describe how that operator would be evaluated if the operator
were left associative.
\end{question}
