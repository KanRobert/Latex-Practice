% ------------------------------------------------------------------------
% file `C++Primer_en-exercises-101-exercise-body.tex'
%
%     exercise of type `exercises' with id `101'
%
% generated by the `exercises' environment of the
%   `xsim' package v0.8a (2017/05/19)
% from source `C++Primer_en' on 2018/03/11 on line 15464
% ------------------------------------------------------------------------
\begin{question}
Assume we have a class named \verb|NoDefault| that has a
constructor that takes an \verb|int|, but has no default constructor. Define a class
\verb|C| that has a member of type \verb|NoDefault|. Define the default constructor for \verb|C|.
\end{question}

\begin{question}
Is the following declaration legal? If not, why not?
\begin{lstlisting}
vector<NoDefault> vec(10);
\end{lstlisting}
\end{question}

\begin{question}
What if we defined the \verb|vector| in the previous execercise to
hold objects of type \verb|C|?
\end{question}

\begin{question}
Which, if any, of the following statements are untrue? Why?
\begin{enumerate}[label=(\alph*)]
^^I\item A class must provide at least one constructor.
^^I\item A default constructor is a constructor with an empty parameter list.
^^I\item If there are no meaningful default values for a class, the class should not
provide a default constructor.
^^I\item If a class does not define a default constructor, the compiler generates
one that initializes each data member to the default value of its associated
type.
\end{enumerate}
\end{question}
