% ------------------------------------------------------------------------
% file `C++Primer_en-exercises-59-exercise-body.tex'
%
%     exercise of type `exercises' with id `59'
%
% generated by the `exercises' environment of the
%   `xsim' package v0.8a (2017/05/19)
% from source `C++Primer_en' on 2017/12/05 on line 9411
% ------------------------------------------------------------------------
\begin{question}
Using an \verb|if–else| statement, write your own version of the
program to generate the letter grade from a numeric grade.
\end{question}

\begin{question}
Rewrite your grading program to use the conditional operator
(\S~\ref{sec:the conditional operator}, p. \pageref{sec:the conditional operator}) in place of the \verb|if–else| statement.
\end{question}

\begin{question}
Correct the errors in each of the following code fragments:
\begin{enumerate}[label=(\alph*)]
^^I\item
\begin{lstlisting}
if (ival1 != ival2)
   ival1 = ival2
else ival1 = ival2 = 0;
\end{lstlisting}

^^I\item
\begin{lstlisting}
if (ival < minval)
   minval = ival;
   occurs = 1;
\end{lstlisting}

^^I\item
\begin{lstlisting}
if (int ival = get_value())
   cout << "ival = " << ival << endl;
if (!ival)
   cout << "ival = 0\n";
\end{lstlisting}

^^I\item
\begin{lstlisting}
if (ival = 0)
ival = get_value();
\end{lstlisting}
\end{enumerate}
\end{question}

\begin{question}
What is a “dangling \verb|else|”? How are \verb|else| clauses resolved in C++?
\end{question}
