% ------------------------------------------------------------------------
% file `C++Primer_en-exercises-126-exercise-body.tex'
%
%     exercise of type `exercises' with id `126'
%
% generated by the `exercises' environment of the
%   `xsim' package v0.8a (2017/05/19)
% from source `C++Primer_en' on 2018/03/11 on line 19607
% ------------------------------------------------------------------------
\begin{question}
Add a \verb|virtual| debug function to your \verb|Quote| class hierarchy
that displays the data members of the respective classes.
\end{question}

\begin{question}
Is it ever useful to declare a member function as both
\verb|override| and \verb|final|? Why or why not?
\end{question}

\begin{question}
Given the following classes, explain each \verb|print| function:
\begin{lstlisting}
class base {
public:
   string name() { return basename; }
   virtual void print(ostream &os) { os << basename; }
private:
   string basename;
};

class derived : public base {
public:
   void print(ostream &os) { print(os); os << " " << i; }
private:
   int i;
};
\end{lstlisting}
If there is a problem in this code, how would you fix it?
\end{question}

\begin{question}
Given the classes from the previous exercise and the
following objects, determine which function is called at run time:
\begin{table}[H]
\begin{tabular}{lll}
\verb|base bobj;| & \verb|base *bp1 = &bobj;| & \verb|base &br1 = bobj;| \\
\verb|derived dobj;| & \verb|base *bp2 = &dobj;| & \verb|base &br2 = dobj;| \\
\end{tabular}
\end{table}
\begin{enumerate}[label=(\alph*)]
^^I\item \verb|bobj.print();|
^^I\item \verb|dobj.print();|
^^I\item \verb|bp1->name();|
^^I\item \verb|bp2->name();|
^^I\item \verb|br1.print();|
^^I\item \verb|br2.print();|
\end{enumerate}
\end{question}
