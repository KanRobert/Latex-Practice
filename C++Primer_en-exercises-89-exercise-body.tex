% ------------------------------------------------------------------------
% file `C++Primer_en-exercises-89-exercise-body.tex'
%
%     exercise of type `exercises' with id `89'
%
% generated by the `exercises' environment of the
%   `xsim' package v0.8a (2017/05/19)
% from source `C++Primer_en' on 2018/03/12 on line 13770
% ------------------------------------------------------------------------
\begin{question}
Define your own versions of the \verb|add|, \verb|read|, and \verb|print|
functions.
\end{question}

\begin{question}
Rewrite the transaction-processing program you wrote for the
exercises in \S~\ref{qst:revise your transaction-processing program using combine and isbn} (p. \pageref{qst:revise your transaction-processing program using combine and isbn}) to use these new functions.
\end{question}

\begin{question}
Why does \verb|read| define its \verb|Sales_data| parameter as a plain
reference and \verb|print| define its parameter as a reference to \verb|const|?
\end{question}

\begin{question}
Add operations to read and print \verb|Person| objects to the code
you wrote for the exercises in \S~\ref{qst:write a class named person} (p. \pageref{qst:write a class named person}).
\end{question}

\begin{question}
What does the condition in the following \verb|if| statement do?
\begin{lstlisting}
if (read(read(cin, data1), data2))
\end{lstlisting}
\end{question}
