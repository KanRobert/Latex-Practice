% ------------------------------------------------------------------------
% file `C++Primer_en-exercises-109-exercise-body.tex'
%
%     exercise of type `exercises' with id `109'
%
% generated by the `exercises' environment of the
%   `xsim' package v0.8a (2017/05/19)
% from source `C++Primer_en' on 2018/03/09 on line 16914
% ------------------------------------------------------------------------
\begin{question}
Use the function you wrote for the first exercise in \S~\ref{qst:function that takes and returns an istream} (p. \pageref{qst:function that takes and returns an istream}) to print the contents of an \verb|istringstream| object.
\end{question}

\begin{question}
Write a program to store each line from a file in a
\verb|vector<string>|. Now use an \verb|istringstream| to read each element from
the \verb|vector| a word at a time.
\end{question}

\begin{question}
The program in this section defined its \verb|istringstream|
object inside the outer \verb|while| loop. What changes would you need to make if
\verb|record| were defined outside that loop? Rewrite the program, moving the
definition of \verb|record| outside the \verb|while|, and see whether you thought of all
the changes that are needed.
\end{question}

\begin{question}
Why didn’t we use in-class initializers in \verb|PersonInfo|?
\end{question}
